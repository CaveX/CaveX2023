\subsection{Existing Prototype}
\label{subsec:existing-prototype}
In this section a detailed analysis of the 2022 CaveX project team's prototype and the conceptual work it was built upon is conducted with a particular focus on the associated software. The purpose of this is to identify factors which influence the accomplishment of the 2023 objectives (see Table \ref{tab:objectives}). Hardware is discussed where it constrains or otherwise impacts the implementation of new software.

\subsubsection{Analysis}
The current prototype functions using two primary software components: Robot Operating System (ROS) version 1 and the Commonwealth Scientific and Industrial Research Organisation's (CSIRO) open-source Syropod High-level Controller (OpenSHC). ROS serves as the backbone of all control and communication-related software and runs on a Linux operating system, Ubuntu 18.04 (Bright et al. 2022). Software within ROS has several layers of abstraction: packages, nodes, topics, and messages. Packages are repositories of executable files (nodes) which interface with ROS and the libraries required to run such nodes. Nodes are typically single-function executables launched by ROS, topics are the means of inter-node communication, and messages are the agreed format by which a topics communication occurs (ROS Wiki 2019a; ROS Wiki 2019b; ROS Wiki 2022). The current prototype runs three nodes to achieve its function: dynamixel\_interface  for interfacing with the servo motors, shc for running the OpenSHC software, and syropod\_remote for listening to controller-based commands. The shc node receives a desired velocity vector from syropod\_remote through the desired\_velocity topic before being used to calculate the desired position of each joint. These positions are then transmitted to dynamixel\_interface via the desired\_joint\_states topic (CSIRO Robotics 2021; CSIRO Robotics 2020). The complexity of the existing control system may make integrating improved algorithms difficult by constraining it to function within the existing code base and, therefore, make achieving OB3 difficult. However, the kinematic model used by OpenSHC contains considerable overlap with many existing hexapod control algorithms (Tam et al. 2020; Zangrandi, Arrigoni \& Braghin 2021). Hence, it is feasible to build from the existing model. 

The prototype's LiDAR sensor is a Velodyne VLP-16 puck (Bright et al 2022). In its current configuration it transmits 1 megabyte (MB) of point data every second. The sensor provides 360\textdegree\space horizontal coverage and a vertical field of view (FOV) of $\pm 15$\textdegree\space relative to the sensor's horizontal plane. Additionally, the data is transmitted directly to the Jetson Nano via 10 gigabit (Gb) per second ethernet (Velodyne 2019). Thus, it is sufficient to support OB1, OB2, and OB4. However, the limited processing power of the Jetson Nano means significant care may be required in the implementation of new algorithms to optimise them and allow room for future development. 

Currently, the prototype does not have a wireless interface with other devices (Bright et al 2022). A lack of WiFi connectivity means it is practically impossible to implement a real-time data monitoring system (OB5). To implement such a system, the prototype must be able to wirelessly transmit data at a high bit rate to a receiver system. Thus a WiFi module must be introduced to achieve OB5.

\subsubsection{Analysis Outcomes}
The analysis of the existing prototype shows that some changes are required to accomplish the 2023 objectives. Specifically, new software needs to be designed carefully to fit in with the existing system and to make the best use of its computational resources (OB1, OB2, OB3, and OB4), and a WiFi module is required to enable the implementation of a realtime system monitoring system (OB5).