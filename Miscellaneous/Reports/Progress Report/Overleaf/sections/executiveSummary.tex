Mapping cave systems is a task characterised by nuanced difficulties but it is essential to understanding such systems. Caves serve as time capsules of Earth itself by shielding matter from the weathering effects of the atmosphere. As such, they play a critical part in deciphering the history of Earth, the solar system it resides within, the civilisations which have existed, and prehistoric creatures that previously roamed the surface. Palaeontologists, geologists, and other researchers often bring digital scanners, such as Light Detection and Ranging (LiDAR) systems, with them on a cave exploration mission to create three-dimensional (3D) maps of them. However, the size, weight, and stationary nature of these systems prevents their use in hard to reach areas and, therefore, limits the maps' completeness. To combat this, the CaveX (Cave eXploration) honours project was initiated with its first iteration in 2021. The aim of the project is to design and build a bio-inspired autonomous robot which can traverse and map cave systems including the hard to reach areas. Dr Elizabeth Reed and Craig Williams from The University of Adelaide are paleontologists who spend considerable time mapping cave systems such as the Naracoorte Caves in Naracoorte, South Australia. Their input has formed the basis for prototype development.

The CaveX project spans multiple years and is currently in its third year. The first iteration, occurring in 2021, saw a bio-inspired hexapod robot designed and built with the ability to map its environment using a 3D LiDAR scanner. Additionally, the recorded LiDAR data was used to perform post-mission Simultaneous Localisation and Mapping (SLAM) to create a map of the explored environment. Additionally, it served as a proof of concept by mapping areas which were previously inaccessible. However, its use was constrained by difficulties with traversing non-uniform terrain and gait control. Subsequently, the 2022 project team assessed the issues with the existing prototype before designing and building an improved prototype. The new prototype maintained the bio-inspired hexapod structure of the 2021 prototype but saw improved servo motors, materials, and software. These improvements resulted in a 15\% weight reduction, improved robustness, and faster movement. The 2023 iteration of the project aims to introduce autonomous functionality into the 2022 prototype to progress towards accomplishing more of the system requirements. Therefore, this iteration's objectives were defined in accordance with the system requirements and thus the user needs. Specifically, they are to implement a real-time localisation and mapping capability, a pathfinding algorithm, terrain sensing and gait control system, obstacle detection and avoidance, and a real-time data monitoring system. In pursuit of such objectives, a systems engineering approach is applied. This approach provides the necessary framework to track user needs and system requirements in relation to accomplishments of the project, thus permitting a detailed justification of user needs being fulfilled. A review of literature exposed a variety of options for accomplishing such fulfilment and, with consideration of the prototype and project's limitations, the most appropriate option for each objective was selected.

Through analysis and testing of the 2022 prototype, its physical and algorithmic limitations were identified. In particular, the prototype struggles to compensate for inclined and declined surfaces. Of the three gaits, tripod was the highest performer in terms of average speed on incline and decline slopes at various angles. This observation is likely due to the additional moving legs supplying a higher total pushing and pulling force depending on whether they are subjected to an incline or decline. Additionally, the wave gait failed to traverse some of the inclines and declines. It was also noted that the robot's feet would slip on the low-friction testing surface. Further, the prototype lacks autonomous functionality and currently has no real time understanding of its environment or its location relative to its environment. The latter is of particular significance as it fundamentally prevents three of the five objectives from being achieved. Thus, while the prototype represents a framework capable of fulfilling the user needs, it currently does not fulfill a key selection of them relating to autonomy. Critically, it was identified that the pursuit of such functionality must make careful considerations to computational optimisation due to the limited processing power of the onboard computer. 

The 2023 CaveX project is currently in the initial phase of the implementation stage. During this stage, the initial algorithm design and implementation is taking place. This is a critical phase in development as multiple future functionalities rely on it. Therefore, it directly impacts the achievement of multiple project objectives and thus the fulfilment of system requirements. Additionally, a major consideration of this stage is the provisioning of hardware and software systems which are a suitable foundation for future iterations of the project. The stage's initial step involves a porting the Fast LiDAR Odometry and Mapping (F-LOAM) algorithm to conduct its primary processing on the robot's graphics processing unit (GPU). While this occurs, the selected dynamic gait optimisation algorithm is integrated with the existing open-source Syropod High-level Controller (OpenSHC). Subsequent to F-LOAM implementation, obstacle detection will be implemented. Finally, since pathfinding depends on an understanding of the obstacles in the environment, it will be implemented after the system has such information. DroneDeploy, the system monitoring platform, will be implemented upon receiving a license for the software. Throughout development, iterative testing and experimentation will occur to ensure the systems are functioning correctly as this allows for rapid debugging in the event of dysfunction. Following this stage is detailed testing. These tests will verify which system requirements have been met and underpin a discussion of the associated challenges in implementation.

The project's completion plan entails algorithm implementation, the associated hardware upgrades, implementation of third-party software for system monitoring and a streamlined development experience, and testing. Algorithm implementation is planned to follow the aforementioned sequence wherein F-LOAM comes first alongside dynamic gait optimisation, then obstacle detection and avoidance, followed by pathfinding, and the development and debugging suite being implemented when it is available. Appropriate testing of each new function will be conducted with the project's objectives, and therefore the system requirements and user needs, in mind. Where testing requires accurate measurement of position, velocity, or acceleration, the OptiTrack system in the Extraterrestrial Environment Simulation (EXTERRES) lab at The University of Adelaide will be used and the relevant data extracted. Completion of this iteration of the project will be marked by fulfillment of all five project objectives. Upon completion, the robot will be able to autonomously navigate complex environments including the Naracoorte Caves. In doing so, it is able to meaningfully contribute to the ongoing mapping and exploration work conducted by Williams, Reed, and other researchers exploring hard to reach areas of an environment.